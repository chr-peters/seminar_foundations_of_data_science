\section{AdaBoost}

\begin{frame}{AdaBoost - How it works}
    \begin{itemize} \pause
        \item AdaBoost invokes the weak learner $T$ times
            on the training data using different sample weights
            $(D_1^{(t)}, ..., D_m^{(t)}), \  t=1, ..., T$ \pause
        \begin{itemize}
            \item The weak learner will minimize the weighted empirical error \pause
        \end{itemize}
        \item In every iteration $t$, the weak learner finds a hypothesis $h_t$ \pause
        \item Then, AdaBoost combines the weak hypotheses $h_t$ like this:
    \end{itemize}
    \begin{equation*}
        h(x) = \text{sign}\left( \sum_{t=1}^T w_t h_t(x) \right)
    \end{equation*} \pause
    Let's look at this in more detail...
\end{frame}

\begin{frame}{One iteration of AdaBoost}
    \begin{enumerate} \pause
        \item Invoke the weak learner on the training data weighted by $D^{(t)}$
        \begin{itemize}
            \item In iteration $t=1$, we use equal weights $D_i^{(t)}=\frac{1}{m}$
        \end{itemize} \pause
        \item Compute a weight for the resulting hypothesis $h_t$ like this:
        \begin{equation*}
            w_t = \frac{1}{2} \text{log} \left( \frac{1}{\epsilon_t} - 1 \right)
        \end{equation*}
        \begin{itemize}
            \item $\epsilon_t$ is the (weighted) training error of $h_t$ \pause
        \end{itemize}
        \item Update the weights $D_i^{(t)}$ like this
        \begin{equation*}
            D_i^{(t+1)} = \frac{D_i^{(t)} \text{exp} \left( -w_t y_i h_t(\mathbf{x}_i) \right)}{
        \sum_{j=1}^m D_j^{(t)} \text{exp} \left( -w_t y_j h_t(\mathbf{x}_j) \right) }
        \end{equation*}
    \end{enumerate}
\end{frame}

\begin{frame}{A step by step example\footnote{Taken from the book \textit{Boosting: Foundations and Algorithms} written by Freund and Schapire~\cite{boosting}.
You can read it for free at \url{https://mitpress.mit.edu/books/boosting}}}
    \begin{figure}
        \includegraphics[width=\textwidth]{img/example_1}
    \end{figure}
\end{frame}

\begin{frame}{A step by step example}
    \begin{figure}
        \includegraphics[width=\textwidth]{img/example_2}
    \end{figure}
\end{frame}

\begin{frame}{A step by step example}
    \begin{figure}
        \includegraphics[width=\textwidth]{img/example_3}
    \end{figure}
\end{frame}

\begin{frame}{A step by step example}
    \begin{figure}
        \includegraphics[width=\textwidth]{img/example_result}
    \end{figure}
\end{frame}