\begin{frame}{What you will know}

    \textbf{$\rightarrow$ The idea behind boosting}
    
    \vspace{0.8cm}
    
    \pause

    \textbf{$\rightarrow$ How to create a strong and efficient learning algorithm}

    \vspace{0.8cm}
    
    \pause

    \textbf{$\rightarrow$ What is AdaBoost and why is it so successful?}
\end{frame}

\section{Let's talk about training a model}

\begin{frame}{How to train a machine learning model}
    \textbf{What we have learned so far...}    
    \begin{itemize}
        \item We have to pick a hypothesis class $\mathcal{H}$ \pause
        \item $\mathcal{H}$ can't be too complex (VC dim needs to be finite) \pause
        \item We need enough training data (more than some threshold $m_\mathcal{H}$) \pause
        \item Then we use ERM to pick the best $h \in \mathcal{H}$ that minimizes the empirical error \pause
    \end{itemize}
    But there is one problem...
\end{frame}

\begin{frame}{The problem with ERM}
\setlength{\fboxrule}{2pt}
    \begin{center}
        \fbox{\huge \textbf{ERM can be hard.}}
    \end{center}

    \pause    
    
    \begin{itemize}
        \item Depending on $\mathcal{H}$, the optimization problem can become arbitrarily complex \pause
        \item e.g. implementing ERM for halfspaces in the non-separable case is computationally hard (chapter 9) \pause
        \item For many interesting classes, it is infeasible to implement ERM
        \begin{itemize}
            \item Solving the optimization problem takes forever
        \end{itemize}
    \end{itemize}

    \pause    
    
    \textbf{...so what can we do?}
\end{frame}

\begin{frame}{A first idea...}
    \textbf{Idea:} Use simpler hypothesis classes where ERM isn't hard.

    \pause    
    
    \begin{itemize}
        \item Problem: Simple classes can be too "weak" to estimate all relationships in the data
        \begin{itemize}
            \item[$\rightarrow$] Can lead to underfitting and poor performance
        \end{itemize} \pause
        \item Approximation error is high ($\rightarrow$ B/C tradeoff) \pause
        \item Still, these classes can be useful for us
        \begin{itemize}
            \item If the resulting hypothesis is at least better than random
        \end{itemize}
    \end{itemize}

    \pause    
    
    Let's call ERM on a simple class a \textbf{weak learner}. We will formally define it later...
\end{frame}

\begin{frame}{The idea behind boosting}
    \textbf{Why not combine many weak learners?}\\
    \textbf{Can this give us an efficient strong learner?}
    
    \pause    
    
    \begin{itemize}
        \item This theoretical question is the origin of boosting \pause
        \item It was first raised in 1988 by Kearns and Valiant~\cite{kv-lbffahf-88} \pause
        \item The first (practical) answer was given in 1995 by Freund and Schapire~\cite{FREUND1997119}
        \begin{itemize}
            \item[$\rightarrow$] It is YES!
        \end{itemize} \pause
        \item The result is \textbf{AdaBoost}, a widely popular and award winning algorithm
        \begin{itemize}
            \item We will take a look at this later...
        \end{itemize}
    \end{itemize}

    \pause    
    
    But first, let's get back to weak learning.
\end{frame}